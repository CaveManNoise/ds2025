\documentclass[12pt]{article}
\usepackage{graphicx}
\usepackage{listings}
\usepackage{caption}

\title{Practical 1:TCP File Transfer: A Simple Client-Server Implementation}
\author{Tran Manh Long-22BI13267}
\date{\today}

\begin{document}

\maketitle

\section{Introduction}
This report show the implementation of File transfer over TCP/IP in CLI that employ the usage of socket library

\section{Protocol Design}
The protocol for this file transfer can be seen below: 

\begin{itemize}
    \item The server listens for a connection from the client on a specified port (port 5000 for example)
    \item The client connects to the server using the server's IP address and port number.
    \item Once the connection is established, the server sends the requested file in chunks of 1024 bytes to the client.
    \item The client receives the data and writes it to a file on its local system usaully on the same path as the program unless specified
    \item The server closes the connection after the file has been fully sent.
    \item If any error occurs during file transfer, the server sends an error message.
\end{itemize}

The protocol is based on the \texttt{TCP/IP} communication model, which ensures reliable, ordered delivery of data.

\section{System Organization}
The system is organized into two main components:
\begin{enumerate}
    \item \textbf{Server}: The server manage the connection, locating the file to send, and send it in chunks to the client.
    \item \textbf{Client}: The client connects to the server, receives the file, and saves it to the local machine.
\end{enumerate}

The server listens on a predefined IP address and port, while the client initiates a connection. Once the connection is established, the server sends the file, and the client writes it to disk.

\section{File Transfer Implementation}
The file transfer is implemented using the Python \texttt{socket} library and execute by using python command on two seperated terminal. Below is a code snippet of the server-side implementation.

\begin{lstlisting}[language=Python, caption=Server-side Implementation]
import socket
import os

def ServerProgram():
    # Server host and port
    host = '127.0.0.1'  # Localhost (use the server IP if running remotely)
    port = 5000

    SeverSocket = socket.socket(socket.AF_INET, socket.SOCK_STREAM)
    SeverSocket.bind((host, port))
    
    SeverSocket.listen(1)#changee the number of connections if needed (i set 1 as default)
    print(f"Server listening on {host}:{port}...")
    conn, address = SeverSocket.accept()
    print(f"Connection from {address} has been established.")
    
    filename = r'E:\distrbuted system\task1\host\transfer_file.txt' 
    if os.path.exists(filename):
        try:
            with open(filename, 'rb') as file:
                print(f"Sending file: {filename}")
                while True:
                    data = file.read(1024)
                    if not data:
                        break  
                    conn.send(data)  
            print(f"File {filename} sent successfully!")
        except Exception as e: #error handling
            print(f"Error sending file: {e}")
            conn.send(b"Error sending file!")
    else:
        print(f"File {filename} not found!")
        conn.send(b"File not found!") 

    conn.close()
    SeverSocket.close()  

if __name__ == '__main__':
    ServerProgram()
\end{lstlisting}

For the client-side implementation, below is the code snippet:

\begin{lstlisting}[language=Python, caption=Client-side Implementation]
import socket

def ClientProgram():
    host = '127.0.0.1' #i set it to transfer in my machine but it can be adjusted to connect any remote host
    port = 5000  # configure the port to the same as host.py
    ClientSocket = socket.socket(socket.AF_INET, socket.SOCK_STREAM)
    ClientSocket.connect((host, port))
    
    filename = 'recievedfile.txt'  #it should save the samepath with the file
    with open(filename, 'wb') as f:
        print(f"Receiving file from the server...")
        while True:
            data = ClientSocket.recv(1024)
            if not data:
                print("No more data received, closing connection.")
                break 
            f.write(data)  

    print(f"I got the file. it will be saved as {filename}")
    ClientSocket.close()

if __name__ == '__main__':
    ClientProgram()

\end{lstlisting}

In the server code, the server listens for a client connection and sends a file in chunks. If the file is not found or an error occurs, the server sends an appropriate error message.

\section{Role of Each Component}

\begin{itemize}
    \item \textbf{Server}:
    \begin{itemize}
        \item Listens for incoming client connections.
        \item Reads the file to be sent.
        \item Sends the file in chunks over the connection.
        \item Handles errors such as file not found.
    \end{itemize}
    
    \item \textbf{Client}:
    \begin{itemize}
        \item Connects to the server.
        \item Receives the file in chunks from the server.
        \item Saves the file locally on the client machine.
    \end{itemize}
\end{itemize}

The server is responsible for managing the file transfer process, while the client initiates the connection and stores the received data.

\end{document}
