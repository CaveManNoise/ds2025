\documentclass[12pt]{article}
\usepackage{amsmath}
\usepackage{listings}
\usepackage{xcolor}
\usepackage{hyperref}
\usepackage{graphicx}
\usepackage{float}

% Settings for code listing
\lstset{
    basicstyle=\ttfamily\small,
    keywordstyle=\color{blue}\bfseries,
    stringstyle=\color{green!60!black},
    commentstyle=\color{gray},
    frame=single,
    numbers=left,
    numberstyle=\tiny\color{gray},
    breaklines=true,
    tabsize=4,
    captionpos=b
}

\title{Hybrid Centralized and Peer-to-Peer Chat System using Sockets}
\author{
    Nguyen Manh Khoi-BI13219 \\
    Nguyen Viet Minh Khoi-BI13221 \\
    Le Quang Huy-BI13190 \\
    Tran Manh Long-BI13267 \\
    Tran Hung Thinh-BI12428 \\
    Tran Hong Nhat-BA12143
}
\date{\today}

\begin{document}

\maketitle

\section{Introduction}
Communication systems are essential in nowaday world. The goal of this project is to build a chat system which can combine the benefits of centralized server and the flexibility of peer-to-peer (P2P) interactions. Centralized models provide easy management and messages broadcasting, while P2P models can make direct communication and reduced server dependency.

\section{Methodology}
The hybrid system using the following methodologies:
\begin{itemize}
    \item \textbf{Centralized Communication}: Users connect to a central server for authentication and broadcasting messages to other connected clients.
    \item \textbf{Peer-to-Peer Communication}: Users can make a direct connections with peers for private messaging, without the central server.
    \item \textbf{Multithreading}: Both the server and clients use multithreading to handle multiple simultaneous connections efficiently.
\end{itemize}

The system architecture consists of two main components:
\begin{enumerate}
    \item \textbf{Central Server}: Manages client connections, broadcasts messages, and provides a list of active users for P2P connections.
    \item \textbf{Clients}: Communicate with the central server and establish P2P connections for private chats.
\end{enumerate}

\begin{figure}[h!]
    \centering
    \includegraphics[width=0.8\textwidth]{architecture.png}
    \caption{System Architecture Diagram}
    \label{fig:architecture}
\end{figure}

\section{Implementation}
The project is implemented in Python, using the socket and threading libraries for network communication. The central server and client functionalities are seperated in respective classes.

\subsection{Central Server}
The central server handles client connections, manages a list of active clients, and broadcasts messages. The following code snippet illustrates the server's core functionality:
\begin{lstlisting}[language=Python, caption=Central Server]
class CentralServer:
    def __init__(self, host='127.0.0.1', port=12345):
        self.server_socket = socket.socket(socket.AF_INET, socket.SOCK_STREAM)  
        self.server_socket.bind((host, port))
        self.server_socket.listen(5)  # Maximum 5 connections
        self.clients = {}  # Dictionary to store client connections

    def handle_client(self, client_socket, client_address):
        try:
            username = client_socket.recv(1024).decode()
            self.clients[username] = client_socket
            print(f"{username} connected from {client_address}")

            while True:
                message = client_socket.recv(1024).decode()
                if message == 'PEER':
                    client_socket.send(str(list(self.clients.keys())).encode())
                else:
                    self.broadcast(message, username)
        except:
            print(f"{username}disconnected.")
            del self.clients[username]
        finally:
            client_socket.close()

    def broadcast(self, message, username):
        for user, conn in self.clients.items():
            if user != username:
                try:
                    conn.send(f"{username}:{message}".encode())
                except:
                    continue

    def start(self):
        print("Central server started...")
        while True:
            client_socket, client_address = self.server_socket.accept()
            threading.Thread(target=self.handle_client, args=(client_socket, client_address)).start()
\end{lstlisting}

\subsection{Peer-to-Peer Client}
Clients communicate with the central server and manage direct connections with peers. Key functionalities include sending messages, receiving messages, and initiating P2P communication.
\begin{lstlisting}[language=Python, caption=Peer-to-Peer Client]
class PeerClient:
    def __init__(self, username, central_host='127.0.0.1', central_port=12345):
        self.username = username
        self.central_socket = socket.socket(socket.AF_INET, socket.SOCK_STREAM)
        self.central_socket.connect((central_host, central_port))
        self.central_socket.send(username.encode())
        self.p2p_port = None  # Placeholder for P2P port

    def receive_messages(self):
        while True:
            try:
                message = self.central_socket.recv(1024).decode()
                print(message)
            except:
                print("Disconnected from server.")
                break

    def send_message(self, message):
        self.central_socket.send(message.encode())

    def peer_to_peer(self, peer_host, peer_port, message=None):
        try:
            peer_socket = socket.socket(socket.AF_INET, socket.SOCK_STREAM)
            peer_socket.connect((peer_host, peer_port))
            if message:
                peer_socket.send(message.encode())
            threading.Thread(target=self.handle_peer, args=(peer_socket,)).start()
        except Exception as e:
            print(f"Error connecting to peer: {e}")
        # Establish P2P connection and send messages

    def start_peer_listener(self, host='127.0.0.1', port=None):
        if port is None:
            port = int(input("Enter port for P2P listener (default is 54321): ") or 54321)
        self.p2p_port = port
        listener_socket = socket.socket(socket.AF_INET, socket.SOCK_STREAM)
        listener_socket.bind((host, port))
        listener_socket.listen(1)
        print(f"Listening for P2P connections on {host}:{port}...")
        while True:
            peer_socket, peer_address = listener_socket.accept()
            print(f"Connected to peer at {peer_address}")
            threading.Thread(target=self.handle_peer, args=(peer_socket,)).start()
        # Listen for incoming P2P connections
\end{lstlisting}
After enter username, cient will be asked for a port for P2P connection(54321 by default). The other client can P2P communicate by this port and IP.

\section{Results}
The hybrid chat system was deployed successfully:
\begin{itemize}
    \item Multiple clients successfully connected to the central server and sent messages.
    \begin{figure}[H]
        \centering
        \includegraphics[width=0.8\textwidth]{client_connection.png}
        \caption{Clients are connected.}
    \end{figure}
    \item Peer-to-peer communication was established, allowing private message between users.
    \begin{figure}[H]
        \centering
        \includegraphics[width=0.8\textwidth]{p2p_communication.png}
        \caption{Private peer-to-peer communication demonstration.}
    \end{figure}
    \item The system demonstrated scalability by handling concurrent client connections efficiently.
    \begin{figure}[H]
        \centering
        \includegraphics[width=0.8\textwidth]{scalability_test.png}
        \caption{Scalability testing with multiple client connections.}
    \end{figure}
\end{itemize}

\section{Conclusion}
This project successfully implemented a hybrid centralized and peer-to-peer chat system using socket.The combine of both models offers flexibility, robustness, and scalability. Future work could adding encryption for secure communication and enhancements for large-scale deployments.


\end{document}
